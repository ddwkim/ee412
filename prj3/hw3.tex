\documentclass{article}
\usepackage{amsmath}
\usepackage{listings}
\usepackage{xcolor}
\usepackage{graphicx}

\definecolor{codegreen}{rgb}{0,0.6,0}
\definecolor{codegray}{rgb}{0.5,0.5,0.5}
\definecolor{codepurple}{rgb}{0.58,0,0.82}
\definecolor{backcolour}{rgb}{0.95,0.95,0.92}

\lstdefinestyle{mystyle}{
    backgroundcolor=\color{backcolour},   
    commentstyle=\color{codegreen},
    keywordstyle=\color{magenta},
    numberstyle=\tiny\color{codegray},
    stringstyle=\color{codepurple},
    basicstyle=\ttfamily\footnotesize,
    breakatwhitespace=false,         
    breaklines=true,                 
    captionpos=b,                    
    keepspaces=true,                 
    numbers=left,                    
    numbersep=5pt,                  
    showspaces=false,                
    showstringspaces=false,
    showtabs=false,                  
    tabsize=2
}

\lstset{style=mystyle}
\begin{document}


\noindent
\textbf{20170082 Dongwon Kim EE412 HW\#3}\\

\noindent
\section*{1}
\subsection*{1-(a)}

The Google Matrix becomes,

\[
A = \begin{bmatrix}
    \frac{1}{3} & \frac{7}{15} & \frac{1}{15} \\ \\
    \frac{1}{3} & \frac{1}{15} & \frac{7}{15} \\ \\
    \frac{1}{3} & \frac{7}{15} & \frac{7}{15}          
\end{bmatrix}
\]

\noindent
Solving using power iteration, we get

\[
\text{PageRank is } \begin{bmatrix} 0.259 & 0.309 & 0.432 \end{bmatrix}
\]

\noindent
The code for the power iteration is as follows.

\begin{lstlisting}[language=Python]
def main():
    adj = [
        [1 / 3, 1 / 3, 1 / 3],
        [1 / 2, 0, 1 / 2],
        [0, 1 / 2, 1 / 2],
    ]
    M = list(map(list, zip(*adj)))
    beta = 0.8
    M = [
        [beta * x + (1 - beta) / len(adj) for x in row]
        for row in M
    ]

    print(M)
    r = [1 / 3, 1 / 3, 1 / 3]
    for _ in range(100):
        r = [
            sum([M[k][j] * r[j] for j in range(len(M))])
            for k in range(len(M))
        ]
    print(r)
    Mr = [
        sum([M[i][j] * r[j] for j in range(len(M))])
        for i in range(len(M))
    ]
    print(Mr)


if __name__ == "__main__":
    main()
\end{lstlisting}


\subsection*{1-(b)}
\begin{enumerate}
    \item [\textbf{(a)}] \[
        \text{PageRank is } \begin{bmatrix} 0.429 & 0.190 & 0.190 & 0.190 \end{bmatrix}
        \]
    \item[\textbf{(b)}] \[
        \text{PageRank is } \begin{bmatrix} 0.386 & 0.171 & 0.271 & 0.171 \end{bmatrix}
        \]
\end{enumerate}

\noindent
The code is as follows.
\begin{lstlisting}[language=Python]
def main():
    adj = [
        [0, 1 / 3, 1 / 3, 1 / 3],
        [1 / 2, 0, 0, 1 / 2],
        [1, 0, 0, 0],
        [0, 1 / 2, 1 / 2, 0],
    ]
    M = list(map(list, zip(*adj)))
    beta = 0.8

    # case (a)
    r = [1 / 4, 1 / 4, 1 / 4, 1 / 4]
    s = [1, 0, 0, 0]
    for _ in range(100):
        r_ = r
        r = [
            sum([M[k][j] * r[j] for j in range(len(M))])
            * beta
            for k in range(len(M))
        ]
        r = [
            r[i] + (1 - beta) * s[i] / sum(s)
            for i in range(len(M))
        ]
    print(r)

    # case (b)
    r = [1 / 4, 1 / 4, 1 / 4, 1 / 4]
    s = [1, 0, 1, 0]
    for _ in range(100):
        r_ = r
        r = [
            sum([M[k][j] * r[j] for j in range(len(M))])
            * beta
            for k in range(len(M))
        ]
        r = [
            r[i] + (1 - beta) * s[i] / sum(s)
            for i in range(len(M))
        ]
    print(r)


if __name__ == "__main__":
    main()
\end{lstlisting}

\noindent
The code performs power iteration on a uniformly initialized vector. The teleport set is given as s.

\section*{2}
\subsection*{2-(a)}
By chain rule,
\[
\frac{\partial L}{\partial w^2_{ij}} = \frac{\partial L}{\partial o_j} \frac{\partial o_j}{\partial z_j} \frac{\partial z_j}{\partial w^2_{ij}}
\]
Since the loss is mean squared error,
\[
\frac{\partial L}{\partial o_j} = (o_j - y_j)
\]
Note
\[
\sigma'(z) = \sigma(z)(1 - \sigma(z))
\]
Therefore, the derivative of the output with respect to $z^2_j$ is
\[
\frac{\partial o_j}{\partial z^2_j} = o_j(1 - o_j)
\]
Denote the hidden layer output as $h_i$. Then, since $z^2_j = \sum_{i=1}^{2} w^2_{ij} h_i$,
\[
\frac{\partial z^2_j}{\partial w^2_{ij}} = h_i
\]
Putting it all together for the second layer weights
\[
\frac{\partial L}{\partial w^2_{ij}} = (o_j - y_j) o_j(1 - o_j) h_i
\]
Gradient of the loss with respect to the weights in the first layer $w^1$ is
\[
\frac{\partial L}{\partial w^1_{ij}} = 
\sum_{k=1}^{2} \left( \frac{\partial L}{\partial o_k} \frac{\partial o_k}{\partial h_j} \frac{\partial h_j}{\partial z^2_j} \frac{\partial z^2_j}{\partial w^1_{ij}} \right)
\]
Since the loss is mean squared error,
\[
\frac{\partial L}{\partial o_k} = (o_k - y_k)
\]
Since $o_k = \sigma(z^2_k)$, $z^2_k = \sum_{i=1}^{2} w^2_{ik} h_i$,
\[
\frac{\partial o_k}{\partial h_i} = w^2_{ik} o_k(1 - o_k)
\]
Since $h_j = \sigma(z^1_j)$,
\[
\frac{\partial h_j}{\partial z^1_j} = h_j(1 - h_j)
\]
Since, $z^1_j = \sum_{i=1}^{2} w^1_{ij} x_i$,
\[
\frac{\partial z^1_j}{\partial w^1_{ij}} = x_i
\]
Putting it all together for the first layer weights
\[
\frac{\partial L}{\partial w^1_{ij}} = \sum_{k=1}^{2} (o_k - y_k) w^2_{ik} o_k(1 - o_k) h_j(1 - h_j) x_i
\]
Where $o_k$ is the output of the second layer and $h_j$ is the output of the first layer,
and $z^k_i$ is the output of the $k$-th layer before the activation function is applied.
In other words, $o_k = \sigma(z^2_k)$ and $h_j = \sigma(z^1_j)$. 

\end{document}